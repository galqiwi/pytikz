
\documentclass[a4paper,12pt]{article}
% math symbols
\usepackage{amssymb,amsmath}
\synctex=1
% for different compilers
\usepackage{ifpdf}
% geometry of page
\usepackage[margin=2cm]{geometry}

% if pdflatex, then
\ifpdf
\usepackage[russian]{babel}
\usepackage[utf8]{inputenc}
\usepackage[unicode]{hyperref}
\usepackage[pdftex]{graphicx}
\usepackage{cmlgc}
% if xelatex, then
\else
% math fonts
\usepackage{fouriernc}
% xelatex specific packages
\usepackage[xetex]{hyperref}
\usepackage{xltxtra}	% \XeLaTeX macro
\usepackage{xunicode}	% some extra unicode support
\defaultfontfeatures{Mapping=tex-text}
\usepackage{polyglossia}	% instead of babel in xelatex
\usepackage{indentfirst}	% 
\setdefaultlanguage{russian}
% fonts
\newfontfamily\cyrillicfont{SchoolBookC}
\newfontfamily\cyrillicfontsf{TextBookC}
\setmonofont{Consolas}
\fi

% several pictures in one figure
\usepackage{subfig}
% calc in TeX expressions
\usepackage{calc}
% nice pictures and plots
\usepackage{pgfplots,tikz,circuitikz}
% different libraries for pictures
\usetikzlibrary{%
  arrows,%
  calc,%
  patterns,%
  decorations.pathreplacing,%
  decorations.pathmorphing,%
  decorations.markings,%
  intersections,%
  decorations.text%
}

\usepackage{tkz-euclide}

\graphicspath{{./pics/}}

% colors of the hyperlinks
\hypersetup{colorlinks,%
  citecolor=blue,%
  urlcolor=blue,%
  linkcolor=red
}

\newcommand{\foot}[1]{\mbox{\footnotesize{#1}}}


\tikzset{
    rotate around with nodes/.style args={#1:#2}{
        rotate around={#1:#2},
        set node rotation={#1},
    },
    rotate with/.style={rotate=\qrrNodeRotation},
    set node rotation/.store in=\qrrNodeRotation,
}


\tikzset{>=latex,%
  interface/.style={postaction={draw,decorate,decoration={border,angle=45,amplitude=0.2cm,segment
        length=1mm}}},%
  spring/.style={thick,decorate,decoration={aspect=0.5, segment
  length=1.3mm, amplitude=1mm,coil}},%
  marrow/.style={postaction={draw,decorate,decoration={markings,
        mark=at position 0.6 with {\arrow{latex}}}}}}

% обозначение угла
\tikzset{arcnode/.style={
            decoration={
                        markings, raise = 3mm,  
                        mark=at position 0.5 with { 
                                    \node[inner sep=0] {#1};
                        }
            },
            postaction={decorate}
      }
}

% команда для отметки угла: проводит дугу между двумя лучами,
% проведёнными между точками #2--#3 и #2--#4
% первый аргумент - необязательный, стиль линии
% #2,#3,#4 - точки
% #5 - радиус дуги для обозначения угла
% #6 - обозначение угла
\newcommand*\marktheangle[6][]{
            \draw[thick,arcnode={#6},#1] let \p2=($(#3)-(#2)$),%
                        \p3=($(#4)-(#2)$),%
                        \n2 = {atan2(\x2,\y2)},%
                        \n3 = {atan2(\x3,\y3)}%
                        in ($(\n2:#5)+(#2)$) arc (\n2:\n3:#5);
}


\pgfplotsset{compat=newest}

\usepackage{expl3}[2012-07-08]
\ExplSyntaxOn
\cs_new_eq:NN \fpeval \fp_eval:n
\ExplSyntaxOff
\usetikzlibrary{calc}

\usetikzlibrary{calc,patterns,angles,quotes}

\usepackage{import}

